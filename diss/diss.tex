\documentclass{article}
\usepackage[parfill]{parskip}
\begin{document}

\section{Abstract}
AAAAA

\tableofcontents
\section{Introduction}
Over the past 50 years, UNIX and its descendent Operating Systems have created a set of interchangeable, near-universal patterns for how computers interact with themselves, each other, and the outside world.
These patterns; files, threads, device drivers, etc. are so powerful that computing devices are classified first by whether or not they are capable of supporting them.
Microprocessors typically support multitasking, networking, and are connected to large amounts of volatile and non-volatile storage.
Microcontrollers, on the other hand, tend to be realtime, with primitive networking and EEPROM storage that often cannot be rewritten at runtime.

As moore's law slows and the graph of computing power over time begins to flatten, semiconductor manufacturers have created new devices.
Because waiting for cheaper hardware is no longer a valid 'optimization' stratergy, manufacturers have begun adding microprocessor-like capabilities to microcontrollers.
These are traditionally classified as microcontrollers, but have capabilities which blur the line between the two classes of device. 
One such super-microcontroller is the Espressif systems ESP-32. A comparison between the ESP-32, Arduino Uno (with an ATmega328P processor), and Pixel 3A (with a Snapdragon 670 processor) follows;

\begin{table}[h]
\begin{tabular}{l|l|l|l}
					& Arduino Uno		& ESP-32		& Pixel 3a		\\ \hline
Clock Frequency		& 16 MHz			& 240 MHz		& 1700 MHz		\\ \hline
RAM					& 32 KB				& 520 KB		& 4000000 KB	\\ \hline
Networking support	& N 				& \textbf{Y}	& \textbf{Y}	\\ \hline
Rewritable storage	& N 				& \textbf{Y}	& \textbf{Y}	\\ \hline
Bluetooth			& N 				& \textbf{Y}	& \textbf{Y}	\\ \hline
Crypto engine		& N 				& \textbf{Y}	& \textbf{Y}	\\ \hline
Ethernet			& N 				& \textbf{Y}	& N				\\ \hline
IO pins				& \textbf{Y}		& \textbf{Y}	& N				\\ \hline

\end{tabular}
\end{table}

\pagebreak
\emph{Note: Many of the features missing from the Arduino can be added using external boards or 'shields'}

The goal of this project is to the first of a new set of patterns designed from the ground up for the new hardware.

\pagebreak
\section{Literature Review}
BBBBB

\section{Implementation}
BBBBB

\section{Results}
BBBBB

\section{Analysis and conclusion}
BBBBB

\section{Reflection}
BBBBB

\end{document}